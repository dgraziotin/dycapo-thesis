%% LyX 1.6.7 created this file.  For more info, see http://www.lyx.org/.
%% Do not edit unless you really know what you are doing.
\documentclass[english]{article}
\usepackage[T1]{fontenc}
\usepackage[latin9]{inputenc}
\usepackage{amstext}
\usepackage{babel}

\begin{document}
\thispagestyle{empty}

\begin{center}
{\LARGE Free University of Bolzano/Bozen }
\par\end{center}{\LARGE \par}

\begin{center}
{\LARGE Faculty of Computer Science}
\par\end{center}{\LARGE \par}

\begin{center}
Thesis
\par\end{center}


\title{Dycapo: On the creation of an open-source Server and a Protocol for
Dynamic Carpooling}

\maketitle
\begin{center}
Daniel Graziotin
\par\end{center}

\begin{center}
Submitted in partial fulfillment of the requirements for the degree
of Bachelor in Applied Computer Science at the Free University of
Bolzano/Bozen
\par\end{center}

\begin{center}
Thesis advisor: Dr. Paolo Massa
\par\end{center}

\begin{center}
October 8, 2010
\par\end{center}

\pagebreak{}

\thispagestyle{empty}


\part*{Abstract}

Carpooling occurs when a driver share his/her private car with one
or more passengers. The benefits of carpooling, also called ridesharing,
are environmental, economical and social. Dynamic Carpooling is a
specific type of Carpooling which allows drivers and passengers to
find suitable lifts close to their desired departure time and directly
on streets. This dissertation describes Dycapo, an open-source system
to provide Dynamic Carpooling services. After a review of the state
of the art, the two main \textquotedbl{}components\textquotedbl{}
are described, namely the protocol and the server architecture. Dycapo
Protocol is an open REST protocol for sharing trip information among
dynamic transit services, taking inspiration from OpenTrip, a previously
proposed protocol. Dycapo Server is a prototype providing a Web Service
for Dynamic Carpooling functionalities, implementing Dycapo Protocol.
Our aim with the release of an open protocol and open source code
is to provide a missing standard and platform that providers of Dynamic
Carpooling services can adopt and extend.

\pagebreak{}

\thispagestyle{empty}


\part*{Riassunto}

\pagebreak{}

\thispagestyle{empty}


\part*{Kurzfassung}

\pagebreak{}

\thispagestyle{empty}

\tableofcontents{}

\pagebreak{}

\thispagestyle{empty}

\listoffigures


\listoftables


\pagebreak{}

\pagenumbering{arabic}


\section{Introduction}


\subsection{General Information }

Private car travelling is an efficient and wasteful way of transportation.
Most cars are occupied by just one or two people. Average car occupancy
in the U.K. is reported to be 1.59 persons/car, in Germany only 1.05\cite{key-0}.
Such an inefficient transportation system causes problems such as
a waste of resources, in terms of gasoline and time. It causes pollutes
in the air. It stresses the parking. One solution is carpooling, i.e.
the share of a private vehicle between one or more passenger. 


\subsection{Dynamic Carpooling}

 \pagebreak{}


\section{State of the art }

\pagebreak{}


\section{Dycapo System}


\subsection{Protocol}


\subsection{Server}

\pagebreak{}


\section{Conclusions}



\pagebreak{}

\pagebreak{}

\appendix

\section{Appendix: Questionnaire}

\pagebreak{}

\thispagestyle{empty}
\begin{thebibliography}{1}
\bibitem[1]{key-0}Stephan Hartwig at al, {}``Empty Seats Traveling'',
Nokia Research Center, February 14$^{\text{th}}$2007

\end{thebibliography}
\pagebreak{}

\thispagestyle{empty}
\end{document}
